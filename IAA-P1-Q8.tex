\\T_{multA} = 2*T(n/2) + c \\
\text{Logo}, a = 2, b = 2, f(n) = \Theta(1) \\
=> f(n) = O(n^{(\log_{2} 2) - 1}) = O(n^{0}) = O(1) \\
\text{Portanto, pelo teorema mestre}, T(n) = \Theta(n^{\log_{2}2}) = \Theta(n) \\
\newline
T_{multB} = T(n/2) + c \\
\text{Logo}, a = 1, b = 2, f(n) = \Theta(1) \\
=> f(n) = \Theta(n^{\log_{2} 1}) = \Theta(n^{0}) = \Theta(1) \\
\text{Portanto, pelo teorema mestre}, T(n) = \Theta(n^{\log_{2}1} * \log{n}) = \Theta(\log{n})\\
\newline
T_{multC} = T(n-1)+c\\
\text{Passo base:} T(1) = c\\
T(2) = T(1) + c\\
T(2) = 2*c\\
T(n) = T(n-1) + c\\
T(n) = c*(n-1) + c\\
T(n) = c*n\\
\text{Portanto}, T(n) = \Theta(n)\\
\newline
\text{A funcao multB tem o menor tempo de execucao}\\
\text{Prova:}\\
T_{multB} = \Theta(\log{n})\\
=> T_{multB} = o(n)\\
\newline
\text{Ao mesmo tempo que:}\\
T_{multA} = \Theta{n} = T_{multC}
=> T_{multA} = O(n) = T_{multC}
\newline
\text{Como sabemos que se}\\
f(n) = o(g(n)) \wedge h(n) = O(g(n))\\
=> f(n) < h(n)\\
\text{provamos que}\\
T_{multB} < T_{multA} = T_{multC}
